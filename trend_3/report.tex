\documentclass{article}
\usepackage{graphicx,fancyhdr,amsmath,amssymb,amsthm,subfig,url,hyperref,epigraph,lipsum}
\usepackage[margin=1in]{geometry}
\setlength\epigraphwidth{.8\textwidth}

% Bibliography
\usepackage[backend=biber, style=numeric, sorting=none]{biblatex}
\addbibresource{report.bib}

%----------------------- Macros and Definitions --------------------------

\renewcommand{\theenumi}{\bf \Alph{enumi}}

\fancypagestyle{plain}{}
\pagestyle{fancy}
\fancyhf{}
\fancyhead[RO,LE]{\sffamily\bfseries\large University of Delhi}
\fancyhead[LO,RE]{\sffamily\bfseries\large MCS-204 Advanced Computer Networks}
\fancyfoot[RO,LE]{\sffamily\bfseries\thepage}
\renewcommand{\headrulewidth}{1pt}
\renewcommand{\footrulewidth}{1pt}

\graphicspath{{figures/}}

%-------------------------------- Title ----------------------------------

\title{Emerging Trends in Mobile Communications}
\author{
    Samyak Ahuja \\
    \texttt{Class ID: 29}
    \and
    Mayank Kharbanda \\
    \texttt{Class ID: 16}
}

%--------------------------------- Text ----------------------------------

\begin{document}
\maketitle

%--------------------------------- Title ----------------------------------
\section{Li-Fi}

\epigraph{There is a crack in everything.
That's how the light gets in.}{\textcite{selected-poem}}


%---------------------------- Introduction -------------------------------

\subsection{Introduction}

Today, when the world is exploring 5G technology, and smart devices are connecting to the cloud for fluid communication and IOT. 
Daily data traffic is increasing at the steepest rate. {It has been estimated that, mobile data traffic will reach to 77 exabytes per month by 2022. \cite{cisco}} 
It is not hidden from the industry that current transmission system is under pressure.\\


\begin{figure}[!h]
  \includegraphics{res/traffic_trend_li_fi.PNG}
    \caption{Mobile data traffic \cite{cisco}}
  \label{fig:traffic_trend_li_fi}
\end{figure}


To revive the radio-wave spectrum from high traffic concentration, a new technology is attracting scientists, it's Visible light communication using the concept of Orthogonal Frequency Division Multiplexing(OFDM) or simply termed as Li-Fi(Light Fidelity).\newline
Li-Fi was first introduced to the world by Prof. Herald Haas during a TEDGLobal Talk in July 2011.\cite{ted1} 
He showcased the potential of the technology to be integrated in the future communication system. 
From then, scientists from various parts of world started studying for various ways to improve the transmission method.\\


\begin{figure}[!h]
  \includegraphics[width=\linewidth]{res/tech-illustration-li-fi.png}
    \caption{Li-Fi \cite{purelifi}}
  \label{fig:tech-illustration-li-fi}
\end{figure}



The technology uses visible-light spectrum as a medium for data transmission. It comprises of a huge bandwidth of 400THz as compared to radio-waves in GHz.
Moreover, visible light does not have any adverse effect on our body as that of radio-waves. LEDs are perfect candidates for light transmission as they have the property that their intensity can be changed at a very high speed.



\begin{figure}[!h]
  \includegraphics[width=\linewidth]{res/spectrum_li_fi.jpg}
    \caption{spectrum comparison \cite{spectrum}}
  \label{fig:spectrum_li_fi}
\end{figure}


%--------------------------------- Working --------------------------------


\subsection{Working of Li-Fi}


%------------------------ Modulation-Demodulation -------------------------
\subsubsection{Modulation-Demodulation}

{The system uses Orthogonal Frequency Division Multiplexing(OFDM) for modulating the signals. As shown in the figure, first the transmitting data is mapped to complex symbols X(l) by some modulation scheme like M-QAM. 
Then signals are summed using IFFT(Inverse Fast Fourier Transformation). 
Then signal is guarded after P/S conversion and transmitted through light source.\\{\cite{gen-ofdm}}}


{At the receiver's end, the signals are converted from serial to parallel and individual signals are extracted using FFT(Fast Fourier Transformation). Signals are then demodulated and send to the receiver.{\cite{gen-ofdm}}}



\begin{figure}[!h]
  \includegraphics[width=\linewidth]{res/OFDM_li_fi.PNG}
    \caption{General OFDM{\cite{gen-ofdm}}}
  \label{fig:OFDM_li_fi}
\end{figure}


%---------------------------- Hardware -----------------------------
\subsubsection{Hardware Requirements}

{It requires two DSP(Digital Signal Processor) boards, at transmitter and receiver ends respectively, one LED bulb and a Photo-Diode reciever.\\
The Electric Signal from the transmitter are first modulated using OFDM by the DSP board, installed between transmitter and the LED. 
The intensity of the LED to generate the signal is controlled by this DSP.{\cite{hardware}}}\\


{At the other end, Photo-Diode receiver detects the high speed fluctuations of the intensity of LED. DSP connected to Photo-Diode receiver decodes the OFDM signals and transmits it to the receiver.
The fluctuations caused in LED are so fast that it's impossible to detect them by naked eye, and hence serves the purpose of normal LED.{\cite{hardware}}}



\begin{figure}[!h]
  \includegraphics[width=\linewidth]{res/hardware_li_fi.PNG}
    \caption{simple OW system {\cite{hardware}}}
  \label{fig:hardware_li_fi}
\end{figure}



%----------------------------- Advantages -----------------------------

\subsection{Advantages}

\begin{description}
    
    \item [Speed] Data transmission speed can reach as high as 224 gigabits per second under light transmission.

    \item [Bandwidth] The unused bandwidth of 400THz in Visible Light Spectrum can be exploit for transmission. 

    \item [Cost and Availability]  There is no issue of initial setup and availability, the LEDs are much cheaper and can be used in place of fluorescent bulbs easily.
    
    \item [Security]  Light cannot penetrate through walls, and can be localized to the area of operation. Hence, provide secure environment for data transmission.
    
    \item [Efficiency] Energy consumption of LED is much less than other artificial light source and there is not much addition energy required for data transmission making it much efficient.
    
\end{description}


%------------------------ Misconceptions --------------------------------

\subsection{Misconceptions}

\begin{description}
    
    \item [It won't work in dark] As data is transmitted through light, one can think that we have to switch on the LED always, and we cannot keep the room dark. But these LEDs can be dimmed low enough that it will not be visible to human eye and still can be used for transmission.

    \item [It won't work in fog] The PD receiver can detect the mere fluctuations from the light source even if there is fog in-between.

    \item [It's not bidirectional] Li-Fi is a Fully duplex system and networked, hence handover as you move around in space.
    
    \item [Li-Fi doesn't work in sunlight] Li-Fi relies on fast change in light intensity, and not on slowly changing natural sources. Various filters can be used to decrease the interference from other sources.
    
\end{description}

%------------------------ Standardization -----------------------------
\subsection{Global light communication standards{\cite{purelifi}}}

In 2019, IEEE announced formation of 802.11 bb task group which will develop and ratify the Global standard for Li-Fi, opening the doors for the use of technology at global level. 
The team aims to deliver the standards by mid 2021.



\printbibliography

\end{document}
