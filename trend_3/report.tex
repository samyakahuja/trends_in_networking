\documentclass{article}
\usepackage{graphicx,fancyhdr,amsmath,amssymb,amsthm,subfig,url,hyperref,epigraph,lipsum}
\usepackage[margin=1in]{geometry}
\setlength\epigraphwidth{.8\textwidth}

% Bibliography
\usepackage[backend=biber, style=authortitle-comp]{biblatex}
\addbibresource{report.bib}

%----------------------- Macros and Definitions --------------------------

\renewcommand{\theenumi}{\bf \Alph{enumi}}

\fancypagestyle{plain}{}
\pagestyle{fancy}
\fancyhf{}
\fancyhead[RO,LE]{\sffamily\bfseries\large University of Delhi}
\fancyhead[LO,RE]{\sffamily\bfseries\large MCS-204 Advanced Computer Networks}
\fancyfoot[RO,LE]{\sffamily\bfseries\thepage}
\renewcommand{\headrulewidth}{1pt}
\renewcommand{\footrulewidth}{1pt}

\graphicspath{{figures/}}

%-------------------------------- Title ----------------------------------

\title{Emerging Trends in Mobile Communications}
\author{
    Samyak Ahuja \\
    \texttt{Class ID: 29}
    \and
    Mayank Kharbanda \\
    \texttt{Class ID: 16}
}

%--------------------------------- Text ----------------------------------

\begin{document}
\maketitle

\section{Li-Fi}
\epigraph{All men must serve.}{The faceless men, \textcite{dance-with-dragons-11}}
\subsection{Introduction}
Today, when the world is exploring 5G technology, and smart devices are connecting to the cloud for fluid communication and IOT. Daily data traffic is increasing at the steepest rate. {It has been estimated that, mobile data traffic will reach to 77 exabytes per month by 2022.\textcite{cisco}} It is not hidden from the industry that current transmission system is under pressure.\\
To revive the radio-wave spectrum from high traffic concentration, a new technology is attracting scientists, it's Visible light communication using the concept of Orthogonal Frequency Division Multiplexing(OFDM) or simply termed as Li-Fi(Light Fidelity).\newline
Li-Fi was first introduced to the world by Prof. Herald Haas during a TEDGLobal Talk in July 2011. He showcased the potential of the technology to be integrated in the future communication system. From then, scientists from various parts of world started studying for various ways to improve the transmission method.\\
The technology uses visible-light spectrum for data transmission, which comprises of a huge bandwidth of 400THz as compared to radio-waves in GHz. Moreover, visible light does not have any adverse effect on our body as that of radio-waves.

\printbibliography

\end{document}
