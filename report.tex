\documentclass[notitlepage,hidelinks]{article}
\usepackage{graphicx,fancyhdr,amsmath,amssymb,amsthm,subfig,url,hyperref,epigraph,lipsum,wrapfig}
\usepackage[margin=1in]{geometry}
\setlength\epigraphwidth{.8\textwidth}

% Bibliography
\usepackage[backend=biber, style=authortitle-comp, maxcitenames=1, refsection=section]{biblatex}
\addbibresource{report.bib}

%----------------------- Macros and Definitions --------------------------

\fancypagestyle{plain}{}
\pagestyle{fancy}
\fancyhf{}
\fancyhead[RO]{\sffamily\large University of Delhi}
\fancyhead[LO]{\sffamily\large MCS-204 Advanced Computer Networks}
\fancyfoot[RO]{\sffamily\thepage}
\renewcommand{\headrulewidth}{1pt}
\renewcommand{\footrulewidth}{1pt}

\makeatletter
\newcommand*{\toccontents}{\@starttoc{toc}}
\makeatother

%-------------------------------- Title ----------------------------------

\title{Emerging Trends in Mobile Communications}
\author{
    Samyak Ahuja \\
    \texttt{Roll Number - 29}
    \and
    Mayank Kharbanda \\
    \texttt{Roll Number - 16}
}
\date{%
    Group Number - 11\\[2ex]%
    \today
}
%--------------------------------- Text ----------------------------------

\begin{document}

\vspace{0.5in}
\maketitle

\vspace{1in}
\section*{Contents}
\toccontents
\newpage

%--------------------------------TREND-1-----------------------------------

\section{Acoustic data transmission}
\epigraph{ I tried to discover, in the rumor of forests and waves, words
that other men could not hear, and I pricked up my ears to listen to the
revelation of their harmony.}{\textcite{november05}}

\subsection{Introduction}
The rise of IoT devices in the home and workplace has created a world where
data and connectivity are becoming increasingly complex. \textcite{ihs16}
predicts a staggering 75 billion connected devices by 2025, up from 26 billion
in 2019, as shown in Figure \ref{fig:ihs_iot}.  As IoT technology advances and the
demand for efficient ways of communicating data between these devices grows,
the world has witnessed a rise in emerging new data transmission technologies
which are looking to provide secure and effective solutions for sharing
information. One solution to meet these rising new demands is “data-over-sound”.

\begin{figure}[!h]
  \includegraphics[width=\linewidth]{res/iot_market_trend.png}
    \caption{Number of IOT devices that will be installed worldwide from 2019
    to 2025 (in billions).}
  \label{fig:ihs_iot}
\end{figure}


Data-over-sound (DoS) presents a compelling solution for many device-to-device
connectivity applications, particularly for use cases that require
frictionless, low cost connectivity with nearby devices. DoS harnesses devices’
existing speakers and microphones to send and receive data over an acoustic
channel. Because it doesn’t require any additional networking hardware, DoS has
captured the interest of companies interested in adding wireless connectivity
functionality to existing devices. Some big names such as Google and Cisco
already have point-product DoS integrations

There are a number of connectivity technologies available in the market today,
including extremely short range (NFC and QR); short range, high bandwidth
(Bluetooth and Wi-Fi). Each technology has its advantages which makes it more
or less suitable for certain applications. An overview of these considerations
are given in Table \ref{table:feature_dos}.

\begin{table}[!h]
\begin{center}
\begin{tabular}{r|c|c|c|c|c|c|}
\multicolumn{1}{r}{}
 &  \multicolumn{1}{c}{DoS}
 &  \multicolumn{1}{c}{QR}
 &  \multicolumn{1}{c}{NFC}
 &  \multicolumn{1}{c}{Bluetooth}
 &  \multicolumn{1}{c}{Wi-Fi}
 &  \multicolumn{1}{c}{Li-Fi} \\
\cline{2-7}
    Two-way communication & 1 & 0 & 0 & 1 & 1 & 1 \\
\cline{2-7}
    One-to-many broadcast & 1 & 0 & 0 & 0 & 0 & 1 \\ 
\cline{2-7}
    Non line-of-sight communication & 1 & 0 & 0 & 1 & 1 & 0 \\
\cline{2-7}
    Zero pairing/set-up procedure & 1 & 1 & 1 & 0 & 0 & 0 \\
\cline{2-7}
    Low power operation & 1 & 1 & 1 & 1 & 0 & 0 \\
\cline{2-7}
    max data rate & 1 kb/s & 3 kb & 424 kb/s & 25 mb/s & 70 mb/s & 1 gb/s \\
\cline{2-7}
    max range & 100m & 10:1 & 20/cm & 100/cm & 50m & 10m \\
\cline{2-7}
\end{tabular}
\end{center}
\caption{Feature comparison for DoS vs alternative technologies}
\label{table:feature_dos}
\end{table}

\subsection{Advantages of Data-over-Sound}

According to Table \ref{table:feature_dos} following advantages can be listed.

\begin{description}
    \item [Device interoperability] The simple hardware requirement of a
        speaker and/ or microphone make data-over-sound arguably the most
        wide-reaching wireless communication technology in terms of device
        compatibility. Mobile phones, voice controlled devices, and any device
        with an alarm speaker are able to communicate using data-over-sound.
        This includes many legacy devices. Data can also be transmitted over
        media channels such as radio and TV, and over existing telephone lines.

    \item [Frictionless UX]  Data-over-sound requires no pairing or
        configuration, making data transfer as simple as pressing a button. 

    \item [Physically Bounded]  Because sound waves respect room boundaries,
        particularly in the near-ultrasonic range commonly used in
        data-over-sound, transmissions do not pass between neighbouring rooms.
        This means that it can be used for detecting the presence of a device
        with room-level granularity.

    \item [Zero power] The advent of ‘wake-on-sound’ MEMs microphones such as
        the Vesper VM1010 enables devices to communicate using data-over-sound,
        whilst draining virtually no battery power in between communications ($<$
        10 µA). 
\end{description}

\subsection{Applications of DoS}

Based on these advantages, four key use cases for Chirp (a DoS implementation) are emerging, which are
identified here.

\begin{description}
    \item [IoT device provisioning] Configuring a new smart device with network
        credentials and functional configuration remains a disproportionately
        complex task, particularly for headless devices. Chirp offers an
        approach to provisioning which is offline, locally-bounded, and does
        not require any infrastructure modifications. Credentials are encoded
        as audio, optionally layered with cryptography for secure scenarios,
        and broadcast over-the-air to nearby smart devices. \par
        \begin{minipage}{\linewidth}
            \centering
            \includegraphics[width=10cm]{res/provisioning_time.png}
            \captionof{figure}{Time taken to provision a smart device.}
            \label{figure:provisioning_time}
        \end{minipage}

        Figure \ref{figure:provisioning_time} shows the result of a study in
        which some experience of IoT were asked to provision a headless smart
        device using three methods: 

        \begin{enumerate}
            \item Manually, by creating a temporary Wi-Fi hotspot on the
                device, connecting to the hotspot with a web browser, and
                entering credentials; 
            \item Using a Chirp-enabled web interface which asks the user to
                enter credentials and transmits them to the device via
                data-over-sound;
            \item Using a Chirp-enabled mobile app which was able to use the
                device’s existing knowledge of the credentials
        \end{enumerate}
        The Chirp-enabled solutions reduced the provisioning time by 48\% with
        the web-based scenario, and 86\% with the mobile scenario.

    \item [Proximity Detection] A key property of acoustic signals is that
        their propagation respects room boundaries, particularly in the
        near-ultrasonic range. As a consequence, there is a growing uptake for
        near-ultrasonic acoustic beacons for presence sensing at room-level
        granularity

    \item [Two way acoustic NFC] Data-over-sound also supports near-field
        communication scenarios whilst offering a two-way full-duplex mode of
        communication, thus addressing a critical limitation of NFC.  Enabling
        two-way data exchange means that devices can perform
        challenge-and-response dialogues - for example, Diffie-Hellman key
        exchange for secure financial transactions, or securely sending
        un-spoofable receipts to a merchant or buyer. 

    \item [Telemetry in RF-restricted environments] In many sensitive
        environments, RF-based communications are prohibited due to the risk of
        sparks or interference with equipment that pre-dates RF regulation.
        Chirp overcomes these limitations, allowing the industrial IoT to
        harness the benefits of wireless communication without limitation.
\end{description}

\subsection{How it all works?}

The basic idea of data-over-sound is no more complex than a traditional
telephone modem.  Data is encoded into an acoustic signal - either audibly as
“bleeps and tones” inaudibly above human hearing range or “hidden” as
imperceptible modifications of existing speech or music - which is then played
through a medium (typically the air, although it could equally be a wired
telephone line or VoIP stream), and received and demodulated by a ‘listening’
device. The listening device then decodes the acoustic signal and returns the
original data. This process is described in Figure \ref{fig:lifecycle_dos}.

\begin{figure}[!h]
  \includegraphics[width=\linewidth]{res/lifecycle_dos.png}
    \caption{life-cycle of data-over-sound}
  \label{fig:lifecycle_dos}
\end{figure}


As with other data transmission protocols, the entire message can include
specific sequences to indicate the start and expected length of a message, as
well as additional bytes for error detection and correction, as shown in Figure
\ref{fig:dos_message}. Messages are typically encoded using frequency-shift
keying (FSK) modulation, which is robust to background noise and distortion
from acoustic effects such as reverberation. Following extensive testing of a
large range of consumer devices, Chirp considers the ‘safe’ range to be between
1kHz-20kHz. This enables the production of audible messages, or near ultrasonic
when limited to ~17kHz-20kHz, which is imperceptible to most adult humans. 

\begin{figure}[!h]
  \includegraphics[width=\linewidth]{res/dos_message.png}
    \caption{Makeup of the message}
  \label{fig:dos_message}
\end{figure}

All of Chirp’s standard profiles are tested to perform at $>99\%$ reliability
under normal operating conditions. Of the existing data-over-sound solutions
on the market, the main differentiator for Chirp is this reliability and
robustness of the internal decoding technology. This has been demonstrated and
stress-tested in acoustically extreme environments, including nuclear power
stations with upwards of 100dB(A) of background noise, across ranges exceeding
100m.

\nocite{chirp17}
\printbibliography[heading=subbibliography]


\newpage
%--------------------------------TREND-2-----------------------------------
\section{Centralized-RAN}

\epigraph{Technology offers us a unique opportunity, 
though rarely welcome, to practice patience.}
{\parencite{patience12}}

%--------------------------------- Intro ----------------------------------
\subsection{Introduction}

Global mobile data traffic is increasing at a substantial rate. 
It is estimated by \textcite{cisco19} that it will grow seven fold
from 2017 to 2022, with cell network advances and cut off in 
data price. To satisfy the consumer demands the network capacity 
is to be increased. It can be done by adding cell sites or by 
implementing the techniques like Multiple Input Multiple 
Output(MIMO). But increasing cell sites requires high capital 
investment and also results in increase in interference.
\nocite{checko14}


%------------------------------ Architecture -----------------------------
\subsection{Architecture}


%------------------------------ Traditional -------------------------------
\subsubsection{Traditional Macro Base}


In the traditional architecture(Figure \ref{fig:compare_arc_CRAN}a), radio and baseband
processing is done inside a base station. The antenna module 
is generally located near the radio module, coaxial cables
are used to connect them, signal loss associated with them is 
high. This architecture was popular during 1G and 2G mobile 
networks.\nocite{checko14}



%--------------------------------- RRH ----------------------------------
\subsubsection{Base station with RRH}


In the Remote Radio Head (RRH) architecture(Figure \ref{fig:compare_arc_CRAN}b), the base station 
has two components namely, a radio unit and a signal processing 
unit. The radio unit is called a RRH or Remote Radio Unit (RRU).
The signal processing part is called a BBU or Data Unit (DU). 
This architecture was introduced during 3G networks and right now 
the majority of base stations use it.\\


The distance between a RRH and a BBU can be extended up to 40 km,
where the limitation is coming from processing and propagation delay.
In this architecture, the BBU equipment can be placed in a more 
convenient place, enabling cost savings on site rental and 
maintenance compared to the traditional RAN architecture. 
RRHs can be placed up on poles, leveraging efficient cooling and
saving on airconditioning in BBU housing. One BBU can serve many
RRHs. RRHs can be connected to each other in a daisy chained 
architecture. Common Public Radio Interface (CPRI) is the 
radio interface protocol widely used for IQ data transmission
between RRHs and BBUs.(Figure \ref{fig:RRH_BBU_CRAN})\nocite{checko14}



\begin{figure}[!h]
  \includegraphics[width=\linewidth]{res/RRH_BBU_CRAN.PNG}
    \caption{Sub modules of BBU and RRH. Source: \parencite{checko14}}
  \label{fig:RRH_BBU_CRAN}
\end{figure}



\begin{wrapfigure}{l}{0.5\textwidth}
  \includegraphics[scale=0.5]{res/compare_arc_CRAN.PNG}
    \caption{Base station architecture evolution.
    Source: \parencite{checko14}}
  \label{fig:compare_arc_CRAN}
\end{wrapfigure}



%--------------------------------- C-RAN ----------------------------------


\subsubsection{Centralized base station architecture, C-RAN} 

In C-RAN(Figure \ref{fig:compare_arc_CRAN}c), in order to optimize BBU utilization between
heavily and lightly loaded base stations, the BBUs are 
centralized into one entity that is called a BBU Pool.
A BBU Pool acts as a virtualized cluster to perform baseband 
processing. The concept of C-RAN was first introduced by IBM
under the name Wireless Network Cloud (WNC) and builds
on the concept of Distributed Wireless Communication System.
Since then various companies exploited the architecture and 
proposed improvements.\\


Figure \ref{fig:LTE_CRAN} shows an example of a C-RAN mobile LTE
network. The fronthaul part of the network spans from the
RRHs sites to the BBU Pool. The backhaul connects the BBU
Pool with the mobile core network. At a remote site, RRHs are
co-located with the antennas. RRHs are connected to the high
performance processors in the BBU Pool through low latency,
high bandwidth optical transport links. Digital baseband, i.e.,
IQ samples, are sent between a RRH and a BBU.\nocite{checko14}



%------------------------------ Advantages --------------------------------

\subsection{Advantages}\nocite{cmri11}

\begin{description}
    
    \item [Energy Efficient and Cost-saving] With centralized 
    processing, the number of BS sites can be reduced. Thus 
    the air conditioning and other site support equipment's 
    power consumption can be largely reduced.\\ 
    As the BBU pool is a shared resource among a large number
    of virtual BS, it means a much higher utilization rate of
    processing resources and lower power consumption can be
    achieved.
    
    \item [Capacity Improvement] In C-RAN, virtual BS's can work
    together in a large physical BBU pool and they can easily 
    share the signaling, traffic data and channel state 
    information. 
    
    \item [Adaptability to Non-uniform Traffic] C-RAN is also
    suitable for non-uniformly distributed traffic due to the 
    load-balancing capability in the distributed BBU pool.
    
\end{description}


%------------------------------ Challenges --------------------------------

\subsection{Challenges}\nocite{checko14}

\begin{description}
    
    \item [Bandwidth, Latency and Jitter] C-RAN brings a huge
    overhead on optical links between RRH and BBU pool as shown
    in figure \ref{fig:RRH_BBU_CRAN}.
    The transport network not only needs to support high bandwidth 
    and be cost efficient, but also needs to support strict latency 
    and jitter requirements.
    
    \item [BBU cooperation, interconnection and clustering] 
    There is a need to improve the performance between BBUs. 
    Cells should be optimally clustered to be assigned to one
    BBU Pool, in order to achieve statistical multiplexing gain,
    facilitate CoMP, but also to prevent the BBU Pool and the
    transport network from overloading. 
    
    \item [Virtualization technique] A virtualization technique
    needs to be proposed to distribute or group processing between
    virtual base station entities and sharing of resources among 
    multiple operators. Any processing algorithm should be 
    expected to work real time - dynamic processing capacity 
    allocation is necessary to deal with a dynamically changing
    cell load.
    
\end{description}



\begin{wrapfigure}{r}{0.5\textwidth}
  \includegraphics[scale=0.7]{res/LTE_CRAN.PNG}
    \caption{C-RAN LTE mobile network. Source: \parencite{checko14}}
  \label{fig:LTE_CRAN}
\end{wrapfigure}



%--------------------------------- Future --------------------------------


\subsection{C-RANs towards 5G}\nocite{peng16}


It is envisioned that 5G will bring a 1000x increase in terms 
of area capacity compared with 4G, achieve a peak rate in
the range of tens of Gbps, support a roundtrip latency of
about 1 ms as well as connections for a trillion of devices,
and guarantee ultra reliability.\\

The field trials conducted by China Mobile have verified the
throughput gain brought by C-RANs based on an uplink LTE model,
reaching up to near 300\%. Through dense RRHs in C-RANs,
massive connections are efficiently supported, and it is not
hard to provide good service for trillion of devices if the
density of RRHs is sufficiently high. Although a big gap
is still observed compared to 5G requirements, the result
has shown the potential advantages of C-RANs. Meanwhile,
different advanced techniques can be involved in C-RANs to
further improve the spectrum efficiency.

%--------------------------------- Bib ----------------------------------

\printbibliography[heading=subbibliography]




\newpage
%--------------------------------TREND-3-----------------------------------
\section{Li-Fi}

\epigraph{There is a crack in everything.
That's how the light gets in.}{\parencite{selected-poem}}


%---------------------------- Introduction -------------------------------

\subsection{Introduction}

Today, when the world is exploring 5G technology, and smart devices are
connecting to the cloud for fluid communication and IOT.  Daily data traffic is
increasing at the steepest rate. According to \textcite{cisco19} it has been
estimated that, mobile data traffic will reach to 77 exabytes per month by
2022. It is not hidden from the industry that current transmission system is
under pressure.\\

\begin{figure}[!h]
  \includegraphics{res/traffic_trend_li_fi.PNG}
    \caption{Mobile data traffic. Source: \parencite{cisco19}}
  \label{fig:traffic_trend_li_fi}
\end{figure}


To revive the radio-wave spectrum from high traffic concentration, a new
technology is attracting scientists, it's Visible light communication using the
concept of Orthogonal Frequency Division Multiplexing(OFDM) or simply termed as
Li-Fi(Light Fidelity).\newline Li-Fi was first introduced to the world by Prof.
Herald Haas during a TEDGLobal Talk in July 2011. In that talk,
\textcite{hass11}, showcased the potential of the technology to be integrated
in the future communication system. From then, scientists from various parts of
world started studying for various ways to improve the transmission method.\\

\begin{figure}[!h]
  \includegraphics[width=\linewidth]{res/tech-illustration-li-fi.png}
    \caption{Li-Fi. Source: \parencite{purelifi}}
  \label{fig:tech-illustration-li-fi}
\end{figure}


The technology uses visible-light spectrum as a medium for data transmission.
It comprises of a huge bandwidth of 400THz as compared to radio-waves in GHz.
Moreover, visible light does not have any adverse effect on our body as that of
radio-waves. LEDs are perfect candidates for light transmission as they have
the property that their intensity can be changed at a very high speed.


\begin{figure}[!h]
  \includegraphics[width=\linewidth]{res/spectrum_li_fi.jpg}
    \caption{spectrum comparison. Source: \parencite{hass18}}
  \label{fig:spectrum_li_fi}
\end{figure}


%--------------------------------- Working --------------------------------


\subsection{Working of Li-Fi}


%------------------------ Modulation-Demodulation -------------------------
\subsubsection{Modulation-Demodulation}

According to \textcite{dimitrov12}, the system uses Orthogonal Frequency
Division Multiplexing(OFDM) for modulating the signals. As shown in the figure
\ref{fig:OFDM_li_fi}, first the transmitting data is mapped to complex symbols
X(l) by some modulation scheme like M-QAM.  Then signals are summed using
IFFT(Inverse Fast Fourier Transformation).  Then signal is guarded after P/S
conversion and transmitted through light source.

At the receiver's end, the signals are converted from serial to parallel and
individual signals are extracted using FFT(Fast Fourier Transformation).
Signals are then demodulated and send to the receiver.

\begin{figure}[!h]
  \includegraphics[width=\linewidth]{res/OFDM_li_fi.PNG}
    \caption{General OFDM. Source: \parencite{dimitrov12}}
  \label{fig:OFDM_li_fi}
\end{figure}


%---------------------------- Hardware -----------------------------
\subsubsection{Hardware Requirements}

\textcite{elgala09} states that LiFi requires two DSP(Digital Signal Processor)
boards, at transmitter and receiver ends respectively, one LED bulb and a
Photo-Diode reciever.\\ The Electric Signal from the transmitter are first
modulated using OFDM by the DSP board, installed between transmitter and the
LED.  The intensity of the LED to generate the signal is controlled by this
DSP.\\

At the other end, Photo-Diode receiver detects the high speed fluctuations of
the intensity of LED. DSP connected to Photo-Diode receiver decodes the OFDM
signals and transmits it to the receiver.  The fluctuations caused in LED are
so fast that it's impossible to detect them by naked eye, and hence serves the
purpose of normal LED.

\begin{figure}[!h]
  \includegraphics[width=\linewidth]{res/hardware_li_fi.PNG}
    \caption{simple OW system. Source: \parencite{elgala09}}
  \label{fig:hardware_li_fi}
\end{figure}



%----------------------------- Advantages -----------------------------

\subsection{Advantages}

\begin{description}
    
    \item [Speed] Data transmission speed can reach as high as 224 gigabits per
        second under light transmission.

    \item [Bandwidth] The unused bandwidth of 400THz in Visible Light Spectrum
        can be exploit for transmission. 

    \item [Cost and Availability]  There is no issue of initial setup and
        availability, the LEDs are much cheaper and can be used in place of
        fluorescent bulbs easily.
    
    \item [Security]  Light cannot penetrate through walls, and can be
        localized to the area of operation. Hence, provide secure environment
        for data transmission.
    
    \item [Efficiency] Energy consumption of LED is much less than other
        artificial light source and there is not much addition energy required
        for data transmission making it much efficient.
    
\end{description}


%------------------------ Misconceptions --------------------------------

\subsection{Misconceptions}

\begin{description}
    
    \item [It won't work in dark] As data is transmitted through light, one can
        think that we have to switch on the LED always, and we cannot keep the
        room dark. But these LEDs can be dimmed low enough that it will not be
        visible to human eye and still can be used for transmission.

    \item [It won't work in fog] The PD receiver can detect the mere
        fluctuations from the light source even if there is fog in-between.

    \item [It's not bidirectional] Li-Fi is a Fully duplex system and
        networked, hence handover as you move around in space.
    
    \item [Li-Fi doesn't work in sunlight] Li-Fi relies on fast change in light
        intensity, and not on slowly changing natural sources. Various filters
        can be used to decrease the interference from other sources.
    
\end{description}

%------------------------ Standardization -----------------------------
\subsection{Global light communication standards}

In 2019, IEEE announced formation of 802.11 bb task group which will develop
and ratify the Global standard for Li-Fi, opening the doors for the use of
technology at global level. 
The team aims to deliver the standards by mid 2021.


\printbibliography[heading=subbibliography]

\end{document}
